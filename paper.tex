\documentclass[sigconf,authordraft]{acmart}

\AtBeginDocument{%
  \providecommand\BibTeX{{%
    Bib\TeX}}}

\copyrightyear{2025}
\acmYear{2025}
\setcopyright{none}
\acmConference[StudConf '25]{16th Student Conference on Software Engineering and Database Systems}{Summer Term 2025}{Magdeburg, Germany}


\begin{document}

\title{Graph traversal languages for large-scale graph processing on modern hardware}

\author{Pascal Zittlau}
\email{pascal.zittlau@ovgu.de}
\affiliation{
	\institution{Otto-von-Guericke-Universität}
	\city{Magdeburg}
	\state{Saxony-Anhalt}
	\country{Germany}
}

\begin{abstract}
	The increasing scale and complexity of graph-structured data necessitate efficient query processing, especially on modern, heterogeneous hardware architectures.
	This paper surveys the state-of-the-art in graph traversal languages designed for large-scale graph processing, focusing on both Property Graph models and RDF graphs.
	It also explores how standard graph algorithms, such as Shortest Path and PageRank, are expressed, integrated, and optimized within query languages for these data models.
	We examine advanced techniques for optimizing query execution, including parallelization strategies for multi-core CPUs and GPUs, the potential of specialized hardware acceleration using FPGAs, and vectorization via SIMD instructions.
	By synthesizing these findings, we identify key challenges and outline future research directions aimed at enhancing the performance, scalability, and hardware utilization of graph traversal languages in demanding analytical workloads.
\end{abstract}

\begin{CCSXML}
	<ccs2012>
	<concept>
	<concept_id>00000000.0000000.0000000</concept_id>
	<concept_desc>Do Not Use This Code, Generate the Correct Terms for Your Paper</concept_desc>
	<concept_significance>500</concept_significance>
	</concept>
	<concept>
	<concept_id>00000000.00000000.00000000</concept_id>
	<concept_desc>Do Not Use This Code, Generate the Correct Terms for Your Paper</concept_desc>
	<concept_significance>300</concept_significance>
	</concept>
	<concept>
	<concept_id>00000000.00000000.00000000</concept_id>
	<concept_desc>Do Not Use This Code, Generate the Correct Terms for Your Paper</concept_desc>
	<concept_significance>100</concept_significance>
	</concept>
	<concept>
	<concept_id>00000000.00000000.00000000</concept_id>
	<concept_desc>Do Not Use This Code, Generate the Correct Terms for Your Paper</concept_desc>
	<concept_significance>100</concept_significance>
	</concept>
	</ccs2012>
\end{CCSXML}

\ccsdesc[500]{Do Not Use This Code~Generate the Correct Terms for Your Paper}
\ccsdesc[300]{Do Not Use This Code~Generate the Correct Terms for Your Paper}
\ccsdesc{Do Not Use This Code~Generate the Correct Terms for Your Paper}
\ccsdesc[100]{Do Not Use This Code~Generate the Correct Terms for Your Paper}

%%
%% Keywords. The author(s) should pick words that accurately describe
%% the work being presented. Separate the keywords with commas.
\keywords{Do, Not, Us, This, Code, Put, the, Correct, Terms, for,
	Your, Paper}
%% A "teaser" image appears between the author and affiliation
%% information and the body of the document, and typically spans the
%% page.
% \begin{teaserfigure}
% 	\includegraphics[width=\textwidth]{sampleteaser}
% 	\caption{Seattle Mariners at Spring Training, 2010.}
% 	\Description{Enjoying the baseball game from the third-base
% 		seats. Ichiro Suzuki preparing to bat.}
% 	\label{fig:teaser}
% \end{teaserfigure}

\maketitle

\section{Introduction}

\begin{itemize}
	\item graphs are a natural abstraction for modeling data
	\item graphs are becoming more important(social networks, scam detection,...)
	\item modern graphs can have billion or trillion of nodes and edges (to big for one machine)
	\item directly operating on graph is hard -> query language
	\item wide variety of query languages available

	\item access patterns traversing graphs are often random and unpredictable (poor cache locality)

	\item modern computers are different
	\item modern CPUs can have hundreds of cores -> efficient parallelism is needed
	\item SIMD is getting wider
	\item GPUs are more prevalent than ever
	\item FPGAs could also be good
	\item In-memory computing(IMC) is an active research topic

	\item Overview over remaining paper
\end{itemize}

\section{Background}

\subsection{Graph Data Models}

\paragraph{Resource Description Framework (RDF)}
\begin{itemize}
	\item W3C standard
	\item represents information as collection of triples
	\item subject, predicate, object
	\item these naturaly form a directed, labeled graph (subject and object = node,
	      predicate = edge)
	\item edge-centric

	\item strengths: standardization, formal semantics
	\item downside: certain structures can't be directly modeled (i.e. properties on
	      relationships require reification)
\end{itemize}

\paragraph{Labeled Property Graph (LPG)}
\begin{itemize}
	\item represents data using nodes, relationships, properties, labels
	\item nodes are entities and can possess multiple labels and properties
	\item relationships connect pairs of nodes, are directed, have a single label and can have properties
	\item node-centric

	\item strengths: often schemaless/-flexible but can also support rigid
	      schemas, flexible data modeling, intuitive representation
	\item downsides: lack of standardization (less with openCypher)
\end{itemize}

\subsection{Large-Scale Graph Processing}

what things have to be thought about when moving to a cluster of machines?

\begin{itemize}
	\item irregular data access (poor spatial and temporal locality, cache misses)
	\item memory bandwidth limitations (when just doing pointer chasing and comparisons, cpu no longer bottleneck)
	\item scalability and partitioning (optimal partitioning is np-hard, different partitioning strategies may favor different types of queries)
	\item parallelism is hard to exploit due to data dependencies
	\item graphs are not static but change (has to be regarded for partitioning, etc.)
\end{itemize}

What should the system do? Graph traversal queries (friends-of-friends) or global graph algorithms (pagerank)? Probably has to be designed differently for each

\section{Established Graph Traversal Languages}

\subsection{(open-)Cypher}
\begin{itemize}
	\item originally developed for Neo4j, openCypher is an open standard developed from that
	\item declarative, property graphs
	\item ASCII-art-like syntax -> tries to make queries intuitiv (like whiteboard)
	\item common keywords: MATCH, WHERE, RETURN, WITH
	\item fixed/variable length paths
	\item semantics are like pattern matching
	\item query pipeline: parsing, semantic verification, optimizing (logical and physical execution plan), executing the plan
	\item used in many systems: Neo4j, SAP HANA Graph, RedisGraph, Memgraph, AgensGraph, Kùzu

	\item optimization on efficient pattern matching, path traversal algorithms, leveraging parallelism
	\item challenges on incremental views, continuous queries, vector search, temporal extensions
\end{itemize}

\subsection{Gremlin}
\begin{itemize}
	\item language of Apache TinkerPop framework
	\item property graphs
	\item functional, data-flow language
	\item supports procedural and declarative queries
	\item query is sequence of steps evaluated from left to right (traversal-oriented)
	\item each step performs an atomic operation on the stream of data
	\item designed to be embedded within host programming languages

	\item OLTP and OLAP mode
	\item optimization through rewriting sequence of steps before execution
	\item parallel execution is possible with GraphComputer model (RDMA, adaptive parallelism control), GPU acceleration gets explored within BitGraph
\end{itemize}

\subsection{SPARQL}
\begin{itemize}
	\item W3C query language for RDF graphs
	\item declarative, based on graph pattern matching
	\item similar syntax to SQL (SELECT, WHERE, FILTER, OPTIONAL)
	\item the core is the basic graph pattern (BGP, set of triple patterns) with variables allowed at any place
	\item semantics is finding all subgraphs that match the BGP, while binding the variables in the patterns to the RDF terms from the data
	\item special case is the possibility of handling non-explicit triples, that can be inferred based on ontologies
	\item SPARQL 1.1 allows specifying regular expressions over the edge label, enabling easier reachability queries

	\item query processing is finding efficient execution plans for the joins required to match a BGP
	\item specialized indexing structures for triples
	\item techniques for handling RDF reasoning
	\item vectorization and parallel processing are desirable
\end{itemize}

\subsection{Comparative Analysis}
\begin{itemize}
	\item show similarities and differences
	\item procedural vs declarative for large/small queries
	\item table, for easy comparison (data model, paradigm, turing-complete, extensibility, typical optimization focus)
\end{itemize}

\section{Emerging Standards}

\subsection{ISO/IEC GQL}
\begin{itemize}
	\item new standard in April 2024 (ISO/IEC 39075)
	\item first new ISO db language since SQL itself
	\item supports querying, DDL, DML
	\item aims to fuse proven concepts from existing graph query languages (openCypher, PGQL, GSQL, etc.) while aligning with SQL
	\item ASCII-art like Cypher
	\item schema-flexible and -constrained
	\item SQL compatible data types
	\item defines a transaction model
\end{itemize}

\subsection{SQL/PGQ}
\begin{itemize}
	\item goal is to allow graph queries within SQL while leveraging existing RDMS
	\item defines syntax for LPGs on top of relational tables
	\item focuses on read-only graph queries
\end{itemize}

\section{Accelaring Traversal on Modern Hardware}
\subsection{Parallel Processing on Multi-Core CPUs}
\begin{itemize}
	\item parallel, simd
	\item bulk synchronous parallel (BSP), asynchronous models
	\item load balancing between cores, minimizing synchronization and communication
\end{itemize}

\subsection{Vectorization (SIMD)}
\begin{itemize}
	\item manual or auto vectorization
	\item AoS vs SoA
	\item data quantization techniques can increase performance even more
	\item challenges are branches -> masking/branchless code
\end{itemize}

\subsection{GPU Acceleration}
\begin{itemize}
	\item GPUs are good for data-intensive things like graph processing
	\item offloading computationally intensive parts of a query or entire algorithms to the GPU
	\item potentially partitioning is needed

	\item hard to program effectively (diverging execution paths within warps, uncoalesced memory accesses, mapping algorithms on MPP, PCIe throughput vs latency)
	\item but if done can be highly effective
\end{itemize}

\subsection{FPGA Acceleration}
\begin{itemize}
	\item allows for designing custom hardware for specific applications
	\item neighbor list traversal, property filtering, pattern matching
	\item graph partitioning and data placement strategies are necessary
	\item specialization is also hard when different kinds of queries need to be processed
	\item potential for extreme performance and energy
\end{itemize}

\subsection*{Summary}
\begin{itemize}
	\item no hardware platform is universally the best -> combination should be chosen
	\item explain, when to choose which (latency -> cpu, throughput -> gpu, ...)
	\item comparison table (pros, cons, key techniques, example, applicability to graph traversal)
\end{itemize}

\section{Advanced Query Processing Optimizations}
optional (probably to much, if included; Or just mention in other parts)
\begin{itemize}
	\item factorization
	\item morsel-driven-parallelism
	\item worst-case optimal joins (WCOJ)
	\item GraphBLAS
\end{itemize}

\section{Graph Algorithms via Traversal Languages}
\begin{itemize}
	\item How do Cypher, Gremlin and SPARQL support common graph operations?
	\item shortest path, PageRank
\end{itemize}

\section{State-of-the-Art, Challenges, and Future}
\begin{itemize}
	% SOTA
	\item What languages seem promising? What abstractions do they provide?
	\item How can common graph algorithms be modeled in these languages?
	\item What hardware is used and explored?

	      % Challenges
	\item What Query Optimization techniques haven't gone from theory to practice?
	\item How to manage Scalability Limits?
	\item How to handle dynamic graphs and schema evolution?
	\item Standardized Benchmarking?

	      % Future
	\item Learded Query Optimization
	\item Adaptive/Dynamic Query Processing
	\item Hardware-Software Co-design and Unified Execution Runtimes
\end{itemize}

\end{document}
